\documentclass[11pt]{article}
\usepackage[utf8]{inputenc}
\usepackage{geometry}
\geometry{a4paper, margin=1in}
\usepackage{enumitem}
\usepackage{hyperref}
\usepackage{titlesec}
\usepackage{parskip}
\usepackage{lmodern}
\usepackage{setspace}

\titleformat{\section}{\bfseries\uppercase}{\thesection}{1em}{}
\titlespacing{\section}{0pt}{*3}{*1.5}

\titleformat{\subsection}{\bfseries}{\thesection}{1em}{}
\titlespacing{\subsection}{0pt}{*2}{*1}

\renewcommand{\familydefault}{\rmdefault} % Use a serif font

\begin{document}

\pagenumbering{gobble} % Remove page numbers
\setstretch{1.1} % Adjust line spacing

\begin{center}
    {\LARGE \textbf{Haobo Zhao}} \\
    \vspace{2mm}
    \small
    Email: \href{mailto:hzhao67@jhu.edu}{hzhao67@jhu.edu} \\
    \vspace{1mm}
    LinkedIn: \href{https://www.linkedin.com/in/haobo-zhao-035529229}{https://www.linkedin.com/in/haobo-zhao-035529229} \\
    \vspace{1mm}
    Website: \href{https://zhbalex.github.io}{https://zhbalex.github.io} \\
\end{center}

\vspace{6mm}

\section*{Education}
\hrule

\noindent\textbf{Johns Hopkins University}, Baltimore, MD \hfill \textit{Sep. 2023 - Present} \\
Master in Mechanical Engineering \\
GPA: 4.0/4.3 \\
Advisor: Dr. Rajat Mittal and Dr. Jung-Hee Seo

\vspace{2mm}

\noindent\textbf{Southern Illinois University}, Carbondale, IL, U.S. \hfill \textit{2022-2023} \\
Aviation Technologies (Dual Degree Program with SAU) \\
GPA: 4.0/4.0\\
Dean's list: Spring 2022, Fall 2022\\
Magna cum Laude


\vspace{2mm}

\noindent\textbf{Shenyang Aerospace University}, Shenyang, Liaoning, China \hfill \textit{2019-2023} \\
Aircraft Propulsion Engineering \\
GPA: 3.8/4.0 \\
National Scholarship (2021, top 1\% in Department) \\
SAU First Class Scholarship (Fall 2020, Fall 2021, Spring 2022)


\vspace{6mm}

\section*{Major Honors and Awards}
\hrule

\begin{itemize}[leftmargin=*,itemsep=1pt]
    \item \textbf{National Scholarship (2021)}: Top 1\% in Department (Academic Performance). 
    China top scholarship among college students
    % \item \textbf{SAU First-Class Scholarship (2020, 2021)}: Top 2\% in Class (Academic Performance + Competition)
    % \item \textbf{SIU Dean's List (2022)}: Excellent Academic Performance
    \item \textbf{First Prize of National Mathematics Competition (China, 2020)}: Top 8\% for non-Mathematical majored college students.
    \item \textbf{Third Prize of Mechanics Competition of Zhou Peiyuan(China, 2021)}: Mechanics competition 
    \item \textbf{Top 5 in China of iCAN Innovation Contest (2021)}: 5/3000 in China, AI video surveillance clarity process
\end{itemize}

\vspace{6mm}

\section*{Research Interests}
\hrule

\noindent Fluid dynamics, applied mechanics, computational fluid dynamics, multiphase flows, biological flows, immersed boundary methods, multi-physics modeling

\vspace{6mm}

\section*{Research Experience}

\noindent\textbf{Johns Hopkins University}, Baltimore, MD \hfill \textit{Sep. 2020 - Present} \\
Ph.D. Thesis (Advisor: Rajat Mittal) – Department of Mechanical Engineering
\begin{itemize}[leftmargin=*,itemsep=1pt]
    \item Developed an imaging data-based computational model of the stomach using an immersed boundary solver.
    \item Incorporated chemical reactions of food hydrolysis into the model to run chemo-fluid dynamic simulations.
    \item Employed particle resolved simulations with 6-degrees-of-freedom to model tablets and large food particles.
    \item Implemented Lagrangian Point Particle Model (LPPM) to simulate small-sized food particles.
    \item Built a pipeline for patient-specific stomach models using cine-MRI data.
    \item Used Volume of Fluids (VOF) approach to account for different density fluids.
    \item Incorporated the stomach muscles via Fluid-Structure-Electrophysiological-Interaction (FSEI) model.
    \item Studied the effect of posture and motility disorders on the dissolution of an oral tablet.
    \item Quantified the effects of motility disorders on the mixing and hydrolysis function of the stomach.
    \item Modeled the consequences of pyloric surgery in different emptying rate disorder patients.
    \item Simulated the mechanism of gastritis due to bile reflux in patients.
\end{itemize}

\vspace{4mm}

\noindent\textbf{Sterlite Technologies}, Dadra \& Nagar Haveli, India \hfill \textit{Jun. 2019 - Sep. 2020} \\
Modeling and Simulations Division – Research \& Development
\begin{itemize}[leftmargin=*,itemsep=1pt]
    \item Developed theoretical and semi-empirical models to predict the onset of signal attenuation in fiber optic cable designs.
    \item Simulated structural deformation in finite-element software to analyze cable designs.
    \item Developed excel-based tools for predicting fiber-optic cable behaviors in extreme temperatures.
    \item Patented grooved cable designs with higher drag to enable air blowing cables to longer distances.
    \item Patented flexible ribbon design to enable more efficient packing of fibers inside the cable.
\end{itemize}

\vspace{4mm}

\noindent\textbf{Indian Institute of Technology}, Kanpur, UP, India \hfill \textit{May 2018 - Jun. 2019} \\
Masters Thesis (Advisor: Arun K. Saha) – Department of Mechanical Engineering
\begin{itemize}[leftmargin=*,itemsep=1pt]
    \item Using a Marker-and-Cell (MAC) based solver, studied the flow past a square cylinder at different blockages.
    \item Developed a code to perform the linear stability analysis of the steady symmetric flow.
    \item Studied the effect of wall proximity on transition from steady to unsteady (Hopf bifurcation) and from symmetric to asymmetric flow (Pitchfork bifurcation).
\end{itemize}

\vspace{4mm}

\noindent\textbf{McGill University}, Quebec, Canada \hfill \textit{May 2017 - Jul. 2017} \\
MITACS Globalink Summer Internship – Department of Mining Engineering \\
Advisor: Agus P. Sasmito
\begin{itemize}[leftmargin=*,itemsep=1pt]
    \item Studied the flow of mine backfill slurry (40-70\% solid fraction) through the hydraulic network in mines.
    \item Modeled the slurry as a two-phase mixture, using Dense Discrete Phase Model (DDPM), and as a Bingham fluid to predict the pumping requirements through different approaches.
    \item Compared the simulations against different friction factor model predictions and against in-situ data.
\end{itemize}

\vspace{4mm}

\noindent\textbf{Indian Institute of Technology}, Kanpur, UP, India \hfill \textit{May 2016 - Jul. 2016} \\
Summer Undergraduate Research (Advisor: Arun K. Saha)
\begin{itemize}[leftmargin=*,itemsep=1pt]
    \item Implemented the immersed boundary method in an existing finite difference flow solver.
\end{itemize}

\vspace{6mm}

\section*{Journal Publications}

\begin{itemize}[leftmargin=*,itemsep=1pt]
    \item \textbf{S. Kuhar and R. Mittal}, “Computational Models of the Fluid Mechanics of the Stomach.” \textit{Journal of the Indian Institute of Science}, (2024).
    \item \textbf{S. Kuhar, J. H. Seo, P. J. Pasricha, and R. Mittal}, “In silico modelling of the effect of pyloric intervention procedures on gastric flow and emptying in a stomach with gastroparesis”, \textit{Journal of the Royal Society Interface}, 21(210), (2024).
    \item \textbf{S. Kuhar, J. H. Lee, J. H. Seo, P. J. Pasricha, and R. Mittal}, “Effect of stomach motility on food hydrolysis and gastric emptying: Insight from computational models”, \textit{Physics of Fluids}, 34(11), 111909, (2022).
    \item \textbf{J. H. Lee, S. Kuhar, J. H. Seo, P. J. Pasricha, and R. Mittal}, “Computational modeling of drug dissolution in the human stomach: Effects of posture and gastroparesis on drug bioavailability”, \textit{Physics of Fluids}, 34(8), 081904, (2022).
    \item \textbf{B. Bharathan, M. McGuinness, S. Kuhar, M. Kermani, F. P. Hassani, and A. P. Sasmito}, “Pressure loss and friction factor in non-Newtonian mine paste backfill: Modelling, loop test and mine field data.” \textit{Powder Technology}, 344, 443–453, (2019).
\end{itemize}

\vspace{6mm}

\section*{Conference Papers}

\begin{itemize}[leftmargin=*,itemsep=1pt]
    \item \textbf{S. Kuhar, J. H. Seo, P. J. Pasricha, and R. Mittal}, “Computational Fluid Dynamics of Digestion Inside the Stomach.” \textit{Proceedings of the 10th International and 50th National Conference on Fluid Mechanics and Fluid Power (FMFP)}, BFM-057 (2023).
\end{itemize}

\vspace{6mm}

\section*{Patents}

\begin{itemize}[leftmargin=*,itemsep=1pt]
    \item \textbf{S. Kuhar, V. Shukla, S. Sharma, and K. Sahoo}, Sterlite Technologies Ltd, 2021. “Ribbed and grooved cable having embedded strength member with water blocking coating”. U.S. Patent Application 17/347,080 (Granted: 2023).
    \item \textbf{H. Kondapalli, S. Sharma, S. Kuhar, A. Nath, V. Shukla, and B. Sarkaar}, Sterlite Technologies Ltd, 2021. “Intermittently bonded optical fibre ribbon with unequal bond and gap lengths”. U.S. Patent Application 17/139,508 (Granted: 2023).
\end{itemize}

\vspace{6mm}

\section*{Conference Presentations}

\begin{itemize}[leftmargin=*,itemsep=1pt]
    \item \textbf{S. Kuhar, A. Menys, J. H. Seo, and R. Mittal}, “Computational modeling of solid food digestion inside the stomach.” APS Division of Fluid Dynamics, L10.00008 (2023).
    \item \textbf{S. Kuhar and R. Mittal}, “Computational modeling of digestion and drug-dissolution inside the stomach.” ReCoVor, 64A, (2023).
    \item \textbf{S. Kuhar, A. Menys, J. H. Seo, and R. Mittal}, “StomachSim: An in-silico model of stomach biomechanics based on patient-specific imaging data.” NeuroGASTRO-2023, (2023).
    \item \textbf{S. Kuhar, J. H. Seo, P. J. Pasricha, and R. Mittal}, “StomachSim: An in-silico simulator of gastric biomechanics with application to pyloroplasty.” American Physiology Summit, 38, 5729950 (2023).
    \item \textbf{S. Kuhar, J. H. Seo, P. J. Pasricha, and R. Mittal}, “StomachSim: An in-silico simulator of gastric biomechanics with application to pyloroplasty.” Digesta Disease Week, 164.6 (2023).
    \item \textbf{S. Kuhar, J. H. Lee, J. H. Seo, P. J. Pasricha, and R. Mittal}, “StomachSim: an in-silico simulator of gastric biomechanics” The Johns Hopkins Department of Medicine \& Whiting School of Engineering Research Retreat, (2023).
    \item \textbf{S. Kuhar, J. H. Lee, J. H. Seo, P. J. Pasricha, and R. Mittal}, “Biofluid dynamics of digestion in the stomach: Insights from computational modeling.” APS Division of Fluid Dynamics, Z07.00004 (2022).
    \item \textbf{S. Kuhar, J. H. Lee, J. H. Seo, P. J. Pasricha, and R. Mittal}, “The Chemo-Fluid Dynamics of Digestion in the Stomach: Insights from Computational Modeling.” CEAFM Burgers Symposium, (2022).
    \item \textbf{S. Kuhar, J. H. Lee, J. H. Seo, P. J. Pasricha, and R. Mittal}, “The Chemo-Fluid Dynamics of Digestion in the Stomach: Insights from Computational Modeling”, APS Division of Fluid Dynamics, T15.00006, (2021).
    \item \textbf{J. H. Lee, S. Kuhar, J. H. Seo, P. J. Pasricha, and R. Mittal}, “The Fluid Dynamics of the Dissolution of an Oral Drug in the Human Stomach”, APS Division of Fluid Dynamics, T15.00007, (2021).
    \item \textbf{S. Kuhar, Arun K. Saha}, “Linear Stability Analysis Of Two-Dimensional Flow Past A Square Cylinder At Different Blockage Ratios”, Research Scholar Day, Association of Mechanical Engineers, IITK, FTS-18, (2019).
\end{itemize}

\vspace{6mm}

\section*{Professional Service}

\noindent\textbf{Engineer, Modeling \& Simulations Division} \hfill \textit{Jul. 2019 - Sep. 2020} \\
Research \& Development, Sterlite Technologies Limited
\begin{itemize}[leftmargin=*,itemsep=1pt]
    \item Worked with multiple teams to oversee the development of new cable designs.
    \item Created and validated models to predict whether the cable designs would pass mechanical and optical tests (tension, crush, thermal cycling, etc.).
    \item Developed easy-to-use excel-based tools to perform calculations for these models to be used by other teams.
    \item Organized ‘Failure Festival’ and regularly released ‘Modeling \& Simulations Newsletter’.
    \item Investigated unexpected cable failures at customer end (e.g., kinking during installation).
\end{itemize}

\vspace{4mm}

\noindent Helped organize the 2023 JHU Summer Internship Program in Mechanical Engineering

\vspace{6mm}

\section*{Volunteering}

\noindent\textbf{Core Team Member} \hfill \textit{Mar. 2016 - Apr. 2017} \\
Institute Counseling Service, IIT Kanpur
\begin{itemize}[leftmargin=*,itemsep=1pt]
    \item Part of 10-member team responsible for campus-wide counseling service activities.
    \item Negotiated with banks to raise 150k INR in scholarships for needy students.
    \item Led a team of 137 student guides during 6-day long orientation program for freshers with a budget of 450k INR.
    \item Worked with professional counselors to aid in providing emotional, mental, and financial support to students.
    \item Hosted sessions aimed at providing academic or career help to students.
\end{itemize}

\vspace{4mm}

\noindent Assisting my PhD advisor with writing funding grants, submitting annual project reports, and peer-reviewing manuscripts in the field of modeling flow inside the stomach and the intestines.

\vspace{6mm}

\section*{Mentoring}

\begin{itemize}[leftmargin=*,itemsep=1pt]
    \item \textbf{Masters Research}: Weixuan Li, Johns Hopkins University \hfill \textit{2023 - Present}
    \item \textbf{Undergraduate Research}: Aditi Gupta, Johns Hopkins University \hfill \textit{May 2023 - Jul. 2023}
\end{itemize}

\vspace{6mm}

\section*{Teaching Experience}

\noindent\textbf{Teaching Assistant at IIT Kanpur}:
\begin{itemize}[leftmargin=*,itemsep=1pt]
    \item Numerical Methods, Fall 2018
    \item Turbulent Flow, Spring 2019
\end{itemize}

\vspace{4mm}

\noindent\textbf{Teaching Assistant at Johns Hopkins University}:
\begin{itemize}[leftmargin=*,itemsep=1pt]
    \item Numerical Methods, Fall 2021 and Fall 2023
    \item Computational Fluid Dynamics, Spring 2022 and Spring 2024
\end{itemize}

\end{document}
